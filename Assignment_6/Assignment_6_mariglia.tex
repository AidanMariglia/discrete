\documentclass[11pt,fleqn]{article}

\usepackage{amsmath}
\usepackage{amssymb}
\usepackage{url}
\usepackage{listings}
\usepackage{color}

\lstset{language=python,basicstyle=\ttfamily,breaklines=true,showspaces=false,showstringspaces=false,breakatwhitespace=true,texcl=true,escapeinside={\%*}{*)}}

\setlength {\topmargin} {-.15in}
\setlength {\textheight} {8.6in}

\renewcommand{\labelenumi}{\theenumi.}
\renewcommand{\labelenumii}{\theenumii.}
\renewcommand{\labelenumiii}{\theenumiii.}
\newcommand{\be}{\begin{enumerate}}
	\newcommand{\ee}{\end{enumerate}}
\newcommand{\bi}{\begin{itemize}}
	\newcommand{\ei}{\end{itemize}}
\newcommand{\bc}{\begin{center}}
	\newcommand{\ec}{\end{center}}
\newcommand{\bsp}{\begin{sloppypar}}
	\newcommand{\esp}{\end{sloppypar}}
\newcommand{\sglsp}{\ }
\newcommand{\dblsp}{\ \ }
\newcommand{\mdot}{\mathrel.}
\newcommand{\mname}[1]{\mbox{\sf #1}}
\newcommand{\ForallApp}{\forall\,}
\newcommand{\ImpliesAlt}{\Rightarrow}
\newenvironment{proof}{\par\noindent{\bf Proof\sglsp}}{\hfill$\Box$}
\newcommand{\pnote}[1]{{\langle \text{#1} \rangle}} 

\begin{document}
	
	\begin{center}
		
		{\large \textbf{COMPSCI/SFWRENG 2FA3}}\\[2mm]
		{\large \textbf{Discrete Mathematics with Applications II}}\\[2mm]
		{\large \textbf{Winter 2021}}\\[8mm]
		{\huge \textbf{Assignment 6}}\\[6mm]
		{\large \textbf{Dr.~William M. Farmer and Dr.~Mehrnoosh Askarpour}}\\[2mm]
		{\large \textbf{McMaster University}}\\[6mm]
		{\large Revised: March 10, 2021}
		
	\end{center}
	
	\medskip
	
	Assignment 6 consists of two problems.  You must write your solutions
	to the problems using LaTeX.
	
	Please submit Assignment~6 as two files,
	\texttt{Assignment\_6\_\emph{YourMacID}.tex} and
	\texttt{Assignment\_6\_\emph{YourMacID}.pdf}, to the Assignment~6
	folder on Avenue under Assessments/Assignments.
	\texttt{\emph{YourMacID}} must be your personal MacID (written without
	capitalization).  The \texttt{Assignment\_6\_\emph{YourMacID}.tex}
	file is a copy of the LaTeX source file for this assignment
	(\texttt{Assignment\_6.tex} found on Avenue under
	Contents/Assignments) with your solution entered after each problem.
	The \texttt{Assignment\_6\_\emph{YourMacID}.pdf} is the PDF output
	produced by executing
	
	\begin{itemize}
		
		\item[] \texttt{pdflatex Assignment\_6\_\emph{YourMacID}}
		
	\end{itemize}
	
	This assignment is due \textbf{Sunday, March 14, 2021 before midnight.}
	You are allow to submit the assignment multiple times, but only the
	last submission will be marked.  \textbf{Late submissions and files
		that are not named exactly as specified above will not be accepted!}
	It is suggested that you submit your preliminary
	\texttt{Assignment\_6\_\emph{YourMacID}.tex} and
	\texttt{Assignment\_6\_\emph{YourMacID}.pdf} files well before the
	deadline so that your mark is not zero if, e.g., your computer fails
	at 11:50 PM on March 14.
	
	\textbf{Although you are allowed to receive help from the
		instructional staff and other students, your submission must be your
		own work.  Copying will be treated as academic dishonesty! If any of
		the ideas used in your submission were obtained from other students
		or sources outside of the lectures and tutorials, you must
		acknowledge where or from whom these ideas were obtained.}
	
	\newpage
	
	\subsection*{Background}
	For any three regular expressions $\alpha$, $\beta$, and $\gamma$, the following properties hold:
	\be
	\item Commutativity of union: $ \alpha + \beta  = \beta + \alpha$
	\item Associativity of union: $ (\alpha + \beta) + \gamma = \alpha + (\beta + \gamma)$
	\item Associativity of concatenation: $(\alpha \beta) \gamma = \alpha (\beta  \gamma)$
	\item Distribution of union: $ \alpha (\beta + \gamma) = \alpha \beta + \alpha  \gamma$ and $  (\beta + \gamma)\alpha =  \beta \alpha +   \gamma \alpha$.
	\item $\emptyset$ is the identity element for union: $\alpha + \emptyset = \alpha$.
	\item $\emptyset$ is the annihilator element for concatenation: $\alpha  \emptyset = \emptyset \alpha = \emptyset$.
	\item $\epsilon$ is the identity element for concatenation: $\alpha  \epsilon = \alpha$.
	\item Idempotence of Kleene star: $\alpha^{**} = \alpha^*$.
	\item Some additional properties of Kleene star:
		\be
			\item $\alpha^* + \epsilon = \alpha^*$
			\item $\alpha^* + \alpha^* = \alpha^*$
			\item $\alpha^*\alpha^* = \alpha^*$
			\item $\alpha\alpha^* + \epsilon = \alpha^*$
			\item $\alpha^*\alpha + \epsilon = \alpha^*$
		\ee
	\ee
	
	\subsection*{Problems}
	
	\be
	
	\item The regular expression $\alpha = (0110 + 01) (10)^*$ can be written in a more simplified way. 
	\be
	\item \textbf{[10 points]} By reasoning based on $L(\alpha)$, define a shorter regular expression equal to $\alpha$. Explain your reasoning.
	\item \textbf{[10 points]} Using the properties explained above, prove that the expression you came up with in (a), is indeed equal to $\alpha$.
	\ee
	
	\bigskip
	
	\textcolor{blue}{\textbf{Aidan Mariglia, mariglia, 14/03/21}}

        \medskip

        \noindent
        

		\emph{1.a.}\\
		The regular expression $\alpha$ matches words such as
		\[ 0110, 01, 01100, 01101, 010, 011, ... \] and so on.
		From this short example of words which match the regular
		expression, we can see that all words must start with 01.
		However, due to the flexible nature of $(10)^*$, all words
		generate by selecting $0110$ from $(0110 + 01)$ can also
		be generated by $01$ and $(10)^*$.
		We can simplify $(0110 + 01)$ into just
		$(01)$ and achieve the same results. After the changes, 
		$\alpha = (01)(10)^*$.
		
		\emph{2.a.}

		\begin{proof}
			\begin{align*}
			  &\phantom{{}=} (0110 + 01)(10)^* &\pnote{LHS}\\
			  &= ((01)(10)(10)^* + (01)(10)^*) &\pnote{Property 4 and Property 3}\\
			  &= ((01)(10)(10)^* + (01)((10)^* + \epsilon)) & \pnote{Property 9.a}\\
			  &= (01)((10)(10)^* + (10)^* + \epsilon) & \pnote{Property 4}\\
			  &= (01)((10)(10)^* + \epsilon + (10)^*) &\pnote{Property 1}\\
			  &= (01)(10^* + 10^*) &\pnote{Property 2 and Property 9.d}\\
			  &= (01)(10)^* &\pnote{Property 9.b and RHS}\\
			\end{align*}
			Q.E.D.
		\end{proof}
\ee
\end{document}


