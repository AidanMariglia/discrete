\documentclass[11pt,fleqn]{article}

\setlength {\topmargin} {-.15in}
\setlength {\textheight} {8.6in}

\usepackage{amsmath}
\usepackage{amssymb}
\usepackage{amsthm}
\usepackage{color}

\renewcommand{\labelenumii}{\theenumii.}

\newcommand{\bsp}{\begin{sloppypar}}
\newcommand{\esp}{\end{sloppypar}}
\newcommand{\mname}[1]{\mbox{\sf #1}}
\newcommand{\tarrow}{\rightarrow}
\newcommand{\pnote}[1]{{\langle \text{#1} \rangle}}
\newcommand{\sB}{\mbox{$\cal B$}}
\newcommand{\sC}{\mbox{$\cal C$}}
\newcommand{\sF}{\mbox{$\cal F$}}
\newcommand{\sP}{\mbox{$\cal P$}}
\newcommand{\sV}{\mbox{$\cal V$}}

\begin{document}

\begin{center}

  {\large \textbf{COMPSCI/SFWRENG 2FA3}}\\[2mm]
  {\large \textbf{Discrete Mathematics with Applications II}}\\[2mm]
  {\large \textbf{Winter 2021}}\\[8mm]
  {\huge \textbf{Extra Credit Assignment 2}}\\[6mm]
  {\large \textbf{Dr.~William M. Farmer}}\\[2mm]
  {\large \textbf{McMaster University}}\\[6mm]
  {\large Revised: February 23, 2021}

\end{center}

\medskip

Extra Credit Assignment 2 consists of one problem.  You must write
your solution to the problem using LaTeX.

\bsp
Please submit Extra Credit Assignment~2 as two files,
\texttt{EC\_Assignment\_2\_\emph{YourMacID}.tex} and
\texttt{EC\_Assignment\_2\_\emph{YourMacID}.pdf}, to the Extra Credit
Assignment~2 folder on Avenue under Assessments/Assignments.
\texttt{\emph{YourMacID}} must be your personal MacID (written without
capitalization).  The \texttt{EC\_Assignment\_2\_\emph{YourMacID}.tex}
file is a copy of the LaTeX source file for this assignment
(\texttt{EC\_Assignment\_2.tex} found on Avenue under
Contents/Assignments) with your solution entered after the problem.
The \texttt{EC\_Assignment\_2\_\emph{YourMacID}.pdf} is the PDF output
produced by executing
\esp

\begin{itemize}

  \item[] \texttt{pdflatex EC\_Assignment\_2\_\emph{YourMacID}}

\end{itemize}

This assignment is due \textbf{Sunday, March 7, 2021 before midnight.}
You are allow to submit the assignment multiple times, but only the
last submission will be marked.  \textbf{Late submissions and files
  that are not named exactly as specified above will not be accepted!}
It is suggested that you submit your preliminary
\texttt{EC\_Assignment\_2\_\emph{YourMacID}.tex} and
\texttt{EC\_Assignment\_2\_\emph{YourMacID}.pdf} files well before the
deadline so that your mark is not zero if, e.g., your computer fails
at 11:50 PM on March 7.

\textbf{Although you are allowed to receive help from the
  instructional staff and other students, your submission must be your
  own work.  Copying will be treated as academic dishonesty! If any of
  the ideas used in your submission were obtained from other students
  or sources outside of the lectures and tutorials, you must
  acknowledge where or from whom these ideas were obtained.}

\newpage

\subsection*{Background}

A set $S$ is \emph{countably infinite} if there is a bijective
function from $S$ to $\mathbb{N}$.  $S$~is \emph{countable} if it is
finite or countably infinite.  


\subsection*{Extra Credit Problem \textbf{[2 bonus points]}}

Consider the following inductive sets:

\begin{enumerate}

  \item \mname{Nat} is the inductive set defined by the following
    constructors:

  \begin{enumerate}

    \item $\mname{Zero} : \mname{Nat}$.

    \item $\mname{Suc} : \mname{Nat} \rightarrow \mname{Nat}$.

  \end{enumerate}

  \item \mname{Var} is the inductive set defined by the following
    constructors:

  \begin{enumerate}

    \item $\mname{X} : \mname{Nat} \rightarrow \mname{Var}$.

    \item $\mname{Y} : \mname{Nat} \rightarrow \mname{Var}$.

  \end{enumerate}

  \item \mname{Term} is the inductive set defined by the following
    constructors:

  \begin{enumerate}

    \item $\mname{V} : \mname{Var} \rightarrow \mname{Term}$.

    \item $\mname{C} : \mname{Term}$.

    \item $\mname{F} : \mname{Term} \rightarrow \mname{Term}$.

  \end{enumerate}

  \item \mname{Form} is the inductive set defined by the following
    constructors:

  \begin{enumerate}

    \item $\mname{Eq} : \mname{Term} \times \mname{Term} \rightarrow \mname{Form}$.

    \item $\mname{Neg} : \mname{Form} \rightarrow \mname{Form}$.

    \item $\mname{Imp} : \mname{Form} \times \mname{Form} \rightarrow \mname{Form}$.

    \item $\mname{All} : \mname{Var} \times \mname{Form} \rightarrow \mname{Form}$.

  \end{enumerate}

\end{enumerate}
%
The members of $\mname{Form}$ represent the formulas in a first-order
language.  Prove that $\mname{Form}$ is countably infinite.  Hint: Use
G\"odel numbering to assign a unique natural number to each member of
$\mname{Form}$.


\bigskip

\noindent
\textcolor{blue}{\textbf{Put your name, MacID, and date here.}}

\bigskip

\noindent
\textcolor{blue}{\textbf{Put your solution here.}}

\end{document}


