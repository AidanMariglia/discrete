\documentclass[11pt,fleqn]{article}

\setlength {\topmargin} {-.15in}
\setlength {\textheight} {8.6in}

\usepackage{amsmath}
\usepackage{amssymb}
\usepackage{amsthm}
\usepackage{url}
\usepackage{color}
\usepackage{tikz}
\usetikzlibrary{automata,positioning,arrows}

\setlength {\topmargin} {-.15in}
\setlength {\textheight} {8.6in}

\renewcommand{\labelenumi}{\theenumi.}
\renewcommand{\labelenumii}{\theenumii.}
\renewcommand{\labelenumiii}{\theenumiii.}
\newcommand{\be}{\begin{enumerate}}
	\newcommand{\ee}{\end{enumerate}}
\newcommand{\bi}{\begin{itemize}}
	\newcommand{\ei}{\end{itemize}}
\newcommand{\bc}{\begin{center}}
	\newcommand{\ec}{\end{center}}
\newcommand{\bsp}{\begin{sloppypar}}
	\newcommand{\esp}{\end{sloppypar}}
\newcommand{\mname}[1]{\mbox{\sf #1}}
\newcommand{\sB}{\mbox{$\cal B$}}
\newcommand{\sC}{\mbox{$\cal C$}}
\newcommand{\sF}{\mbox{$\cal F$}}
\newcommand{\sM}{\mbox{$\cal M$}}
\newcommand{\sP}{\mbox{$\cal P$}}
\newcommand{\sV}{\mbox{$\cal V$}}
\newcommand{\set}[1]{{\{ #1 \}}}
\newcommand{\Neg}{\neg}
\ifdefined \And
\renewcommand{\And}{\wedge}
\else
\newcommand{\And}{\wedge}
\fi
\newcommand{\Or}{\vee}
\newcommand{\Implies}{\Rightarrow}
\newcommand{\Iff}{\LeftRightarrow}
\newcommand{\Forall}{\forall}
\newcommand{\ForallApp}{\forall\,}
\newcommand{\Forsome}{\exists}
\newcommand{\ForsomeApp}{\exists\,}
\newcommand{\mdot}{\mathrel.}
\newcommand{\eps}{\epsilon}
\newcommand{\pnote}[1]{\langle \mbox{#1} \rangle}


\begin{document}

\begin{center}

{\large \textbf{COMPSCI/SFWRENG 2FA3}}\\[2mm]
{\large \textbf{Discrete Mathematics with Applications II}}\\[2mm]
{\large \textbf{Winter 2021}}\\[8mm]
{\huge \textbf{Assignment 9}}\\[6mm]
{\large \textbf{Dr.~William M. Farmer and Dr.~Mehrnoosh Askarpour}}\\[2mm]
{\large \textbf{McMaster University}}\\[6mm]
{\large Revised: March 28, 2021}

\end{center}

\medskip

Assignment 9 consists of two problems.  You must write your solutions
to the problems using LaTeX.

Please submit Assignment~9 as two files,
\texttt{Assignment\_9\_\emph{YourMacID}.tex} and
\texttt{Assignment\_9\_\emph{YourMacID}.pdf}, to the Assignment~9
folder on Avenue under Assessments/Assignments.
\texttt{\emph{YourMacID}} must be your personal MacID (written without
capitalization).  The \texttt{Assignment\_9\_\emph{YourMacID}.tex}
file is a copy of the LaTeX source file for this assignment
(\texttt{Assignment\_9.tex} found on Avenue under
Contents/Assignments) with your solution entered after each problem.
The \texttt{Assignment\_9\_\emph{YourMacID}.pdf} is the PDF output
produced by executing

\begin{itemize}

  \item[] \texttt{pdflatex Assignment\_9\_\emph{YourMacID}}

\end{itemize}

This assignment is due \textbf{Sunday, April 4, 2021 before
  midnight.}  You are allow to submit the assignment multiple times,
but only the last submission will be marked.  \textbf{Late submissions
  and files that are not named exactly as specified above will not be
  accepted!}  It is suggested that you submit your preliminary
\texttt{Assignment\_9\_\emph{YourMacID}.tex} and
\texttt{Assignment\_9\_\emph{YourMacID}.pdf} files well before the
deadline so that your mark is not zero if, e.g., your computer fails
at 11:50 PM on April 4.

\textbf{Although you are allowed to receive help from the
  instructional staff and other students, your submission must be your
  own work.  Copying will be treated as academic dishonesty! If any of
  the ideas used in your submission were obtained from other students
  or sources outside of the lectures and tutorials, you must
  acknowledge where or from whom these ideas were obtained.}

\newpage

\subsection*{Problems}

\be

  \item \textbf{[10 points]} Let $\Sigma = \set{a,b}$ and \[L = \set{x
    \in \Sigma^* \mid \#a(x) \text{ and } \#b(x) \text{ are both
      even}}.\] Construct a total Turing machine that accepts $L$.
    Present the TM using a transition table or diagram formally, and
    describe how it works informally.

  \bigskip

  \textcolor{blue}{\textbf{Aidan Mariglia, mariglia, 4/4/21}}

  $M = (Q, \Sigma, \Gamma, \vdash, \textvisiblespace, \delta, s, t, r)$\\
  $Q = \{s ,q_1 \dots q_4 ,t ,r \} $\\
  $\Sigma = \{a,b\}$\\
  $\Gamma = \Sigma \cup \{\vdash, \textvisiblespace\}$\\
  $\delta$ is defined by the following table:\\

  \begin{table}[h]
    \centering
    \begin{tabular}{lllll}
         & $\vdash$         & a             & b             & $\textvisiblespace$              \\
    s    & $(s, \vdash, R)$ & $(q_2, -, R)$ & $(q_3, -, R)$ & $(t, -, -)$                      \\
    q\_1 & $(-, \vdash, R)$ & $(q_3, -, R)$ & $(q_2, -, R)$ & $(r, -, -)$                      \\
    q\_2 & $(-, \vdash, R)$ & $(q_4, -, R)$ & $(q_1, -, R)$ & $(r, -, -)$                      \\
    q\_3 & $(-, \vdash, R)$ & $(q_1, -, R)$ & $(q_4, -, R)$ & $(r, -, -)$                      \\
    q\_4 & $(-, \vdash, R)$ & $(q_2, -, R)$ & $(q_3, -, R)$ & $(t, -, -)$                      \\
    t    & $(t, \vdash, R)$ & $(t, -, -)$   & $(t, -, -)$   & $(t, -, -)$                      \\
    r    & $(r, \vdash, R)$ & $(r, -, -)$   & $(r, -, -)$   & $(r, -, -)$                     
    \end{tabular}
  \end{table}

  The convention of - indicating any value is accepted is used here as in the lecture slides \medskip
  
  This Turing machine is quite simple. In addition to the start, accept and reject
  states, it makes use of four other states, $q_1$, the state in which \#a(x) and
  \#b(x) are both odd, $q_2$, the state in which only \#a(x) is odd, $q_3$, the
  state in which only \#b(x) is odd, and $q_4$, the state in which neither \#a(x)
  or \#b(x) is odd. The machine reads the input accordingly, and transitions to
  the accept or reject state only when the read head comes across $\textvisiblespace$
  indicating the input string is finished.



  \item \textbf{[10 points]} Let $\Sigma = \set{a,b}$ and \[L = \set{x
    \in \Sigma^* \mid \#a(x) = \#b(x)}.\] Construct a total Turing
    machine that accepts $L$.  Present the TM using a transition table
    or diagram formally, and describe how it works informally.

  \bigskip

  \textcolor{blue}{\textbf{Aidan Mariglia, mariglia, 4/4/21}}
  
  $M = (Q, \Sigma, \Gamma, \vdash, \textvisiblespace, \delta, s, t, r)$\\
  $Q = \{s, q_1 \dots q_3, t, r \}$\\
  $\Sigma = \{a,b\}$\\
  $\Gamma = \Sigma \cup \{\vdash, \textvisiblespace, x \}$\\
  $\delta$ is defined by the following table:\\

  \begin{table}[h]
    \centering
    \begin{tabular}{llllll}
         & $\vdash$         & a             & b             & x           & $\textvisiblespace$ \\
    s    & $(s, \vdash, R)$ & $(q_1, x, R)$ & $(q_2, x, R)$ & $(s,x,R)$   & $(t, -, -)$         \\
    q\_1 & $(-, \vdash, R)$ & $(q_1, a, R)$ & $(q_3, x, L)$ & $(q_1,x,R)$ & $(r, -, -)$         \\
    q\_2 & $(-, \vdash, R)$ & $(q_3, x, L)$ & $(q_2, b, R)$ & $(q_2,x,L)$ & $(r, -, -)$         \\
    q\_3 & $(s, \vdash, R)$ & $(q_3, a, L)$ & $(q_3, b, L)$ & $(q_3,x,L)$ & $(r, -, -)$         \\
    t    & $(t, \vdash, R)$ & $(t, -, -)$   & $(t, -, -)$   & $(r,-,-)$   & $(t, -, -)$         \\
    r    & $(r, \vdash, R)$ & $(r, -, -)$   & $(r, -, -)$   & $(r,-,-)$   & $(r, -, -)$        
    \end{tabular}
  \end{table}

  This Turing machine operates by scanning its tape left to right, deleting
  pairs of a's and b's by replacing them with the placeholder symbol x.
  This ensures that if the read-head can go from the left end marker to the
  empty space to the right of the input without finding any stray a's or b's,
  then the original number of a's and b's will have been equal. After each
  pair is deleted, the read head is reset to the leftmost position before
  attempting another scan of the remaining input on the tape.


\ee

\end{document}


