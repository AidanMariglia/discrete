\documentclass[11pt,fleqn]{article}

\usepackage{amsmath}
\usepackage{amssymb}
\usepackage{url}
\usepackage{listings}
\usepackage{color}

\lstset{language=python,basicstyle=\ttfamily,breaklines=true,showspaces=false,showstringspaces=false,breakatwhitespace=true,texcl=true,escapeinside={\%*}{*)}}

\setlength {\topmargin} {-.15in}
\setlength {\textheight} {8.6in}

\renewcommand{\labelenumi}{\theenumi.}
\renewcommand{\labelenumii}{\theenumii.}
\renewcommand{\labelenumiii}{\theenumiii.}
\newcommand{\be}{\begin{enumerate}}
\newcommand{\ee}{\end{enumerate}}
\newcommand{\bi}{\begin{itemize}}
\newcommand{\ei}{\end{itemize}}
\newcommand{\bc}{\begin{center}}
\newcommand{\ec}{\end{center}}
\newcommand{\bsp}{\begin{sloppypar}}
\newcommand{\esp}{\end{sloppypar}}
\newcommand{\mname}[1]{\mbox{\sf #1}}

\begin{document}

%\thispagestyle{empty}

\bc

  {\large \textbf{COMPSCI/SFWRENG 2FA3}}\\[2mm]
  {\large \textbf{Discrete Mathematics with Applications II}}\\[2mm]
  {\large \textbf{Winter 2021}}\\[8mm]
  {\huge \textbf{Week 02 Exercises}}\\[6mm]
  {\large \textbf{Dr.~William M. Farmer and Dr.~Mehrnoosh Askarpour}}\\[2mm]
  {\large \textbf{McMaster University}}\\[6mm]
  {\large Revised: January 6, 2020}

\ec

\medskip

\subsection*{Background Definitions}

\begin{enumerate}

  \item The notation $\sum^{n}_{i=m}f(i)$ is defined by: 
    \[\sum^{n}_{i=m}f(i) =
      \left\{\begin{array}{ll}
               0                          & \textrm{if } m > n\\
               f(n) + \sum^{n-1}_{i=m}f(i) & \textrm{if } m \le n
             \end{array}
      \right.\] 

  \item The Fibonacci sequence $\mname{fib} : \mathbb{N} \rightarrow
    \mathbb{N}$ is defined by:
    \[\mname{fib}(n) = 
      \left\{\begin{array}{ll}
               0 & \textrm{if } n = 0 \\
               1 & \textrm{if } n = 1 \\
               \mname{fib}(n-1) + \mname{fib}(n-2) & \textrm{if } n \ge 2
             \end{array}
      \right.\]

   \item Let $a,b \in \mathbb{Z}$.  $a$ \emph{divides} $b$, written $a
     \mid b$, if $b = ac$ for some $c \in \mathbb{Z}$.

\end{enumerate}

\subsection*{Exercises}

\be

  \item Prove the following statements:

  
  \be

    \item The sum of two odd integers is an even integer.

    \item If $x$ is an even integer, then $x^2$ is also even.

    \item Let $a,b,c,d \in \mathbb{Z}$.  If $a \mid b$ and $c \mid d$,
      then $ac \mid bd$.   

    \item The square root of 2 is an irrational number.

  \ee

  \item Prove the following statements by weak induction: 

  \be

    \item $\sum^{n}_{i=0}2i = n(n+1)$ for all $n \in \mathbb{N}$.

    \item $\sum^{n}_{i=1}(2i - 1) = n^2$  for all $n \in \mathbb{N}$.

    \item $\sum^{n}_{i=0}i^2 = \frac{n(n+1)(2n + 1)}{6}$ for all
      $n \in \mathbb{N}$.

    \item $\sum^{n-1}_{i=0}2^i = 2^n - 1$ for all $n \in \mathbb{N}$.

    \item $\sum^{n}_{i=0} \mname{fib}(i) = \mname{fib}(n+2) - 1$ for
      $n \in \mathbb{N}$.

    \item $\sum^{n}_{i=0}(\mname{fib}(i))^2 = \mname{fib}(n) *
      \mname{fib}(n+1)$ for all $n \in \mathbb{N}$.

  \ee

  \item Prove the following statements by strong induction:

  \be

    \item If $n \in \mathbb{N}$ with $n \ge 2$, then $n$ is a product
      of prime numbers.

    \item $\mname{fib}(n) < 2^n$ for all $n \in \mathbb{N}$.

    \item It takes $n - 1$ divisions to break up a rectangular
      chocolate bar containing $n$ squares into individual squares.

  \ee


  \item Let $t_n,s_n,o_n$ be the $n$th triangle, square, and oblong
    numbers, respectively, where $n \in \mathbb{N}$.

  \be

    \item Define $t_n,s_n,o_n$ by recursion.

    \item Prove by induction that every triangle number is exactly
      half of an oblong number.

    \item Prove by induction that the sum of every two consecutive
      triangle numbers is a square number.

  \ee

\ee
\end{document}


