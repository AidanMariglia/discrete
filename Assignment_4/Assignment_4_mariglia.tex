\documentclass[11pt,fleqn]{article}

\setlength {\topmargin} {-.15in}
\setlength {\textheight} {8.6in}

\usepackage{amsmath}
\usepackage{amssymb}
\usepackage{amsthm}
\usepackage{url}
\usepackage{color}
\usepackage{tikz}
\usetikzlibrary{automata,positioning,arrows}

\setlength {\topmargin} {-.15in}
\setlength {\textheight} {8.6in}

\renewcommand{\labelenumi}{\theenumi.}
\renewcommand{\labelenumii}{\theenumii.}
\renewcommand{\labelenumiii}{\theenumiii.}
\newcommand{\be}{\begin{enumerate}}
	\newcommand{\ee}{\end{enumerate}}
\newcommand{\bi}{\begin{itemize}}
	\newcommand{\ei}{\end{itemize}}
\newcommand{\bc}{\begin{center}}
	\newcommand{\ec}{\end{center}}
\newcommand{\bsp}{\begin{sloppypar}}
	\newcommand{\esp}{\end{sloppypar}}
\newcommand{\mname}[1]{\mbox{\sf #1}}
\newcommand{\sB}{\mbox{$\cal B$}}
\newcommand{\sC}{\mbox{$\cal C$}}
\newcommand{\sF}{\mbox{$\cal F$}}
\newcommand{\sM}{\mbox{$\cal M$}}
\newcommand{\sP}{\mbox{$\cal P$}}
\newcommand{\sV}{\mbox{$\cal V$}}
\newcommand{\set}[1]{{\{ #1 \}}}
\newcommand{\Neg}{\neg}
\ifdefined \And
\renewcommand{\And}{\wedge}
\else
\newcommand{\And}{\wedge}
\fi
\newcommand{\Or}{\vee}
\newcommand{\Implies}{\Rightarrow}
\newcommand{\Iff}{\LeftRightarrow}
\newcommand{\Forall}{\forall}
\newcommand{\ForallApp}{\forall\,}
\newcommand{\Forsome}{\exists}
\newcommand{\ForsomeApp}{\exists\,}
\newcommand{\mdot}{\mathrel.}
\newcommand{\eps}{\epsilon}
\newcommand{\pnote}[1]{\langle \mbox{#1} \rangle}


\begin{document}

\begin{center}

{\large \textbf{COMPSCI/SFWRENG 2FA3}}\\[2mm]
{\large \textbf{Discrete Mathematics with Applications II}}\\[2mm]
{\large \textbf{Winter 2021}}\\[8mm]
{\huge \textbf{Assignment 4}}\\[6mm]
{\large \textbf{Dr.~William M. Farmer and Dr.~Mehrnoosh Askarpour}}\\[2mm]
{\large \textbf{McMaster University}}\\[6mm]
{\large Revised: February 23, 2021}

\end{center}

\medskip

Assignment 4 consists of two problems.  You must write your solutions
to the problems using LaTeX.

Please submit Assignment~4 as two files,
\texttt{Assignment\_4\_\emph{YourMacID}.tex} and
\texttt{Assignment\_4\_\emph{YourMacID}.pdf}, to the Assignment~4
folder on Avenue under Assessments/Assignments.
\texttt{\emph{YourMacID}} must be your personal MacID (written without
capitalization).  The \texttt{Assignment\_4\_\emph{YourMacID}.tex}
file is a copy of the LaTeX source file for this assignment
(\texttt{Assignment\_4.tex} found on Avenue under
Contents/Assignments) with your solution entered after each problem.
The \texttt{Assignment\_4\_\emph{YourMacID}.pdf} is the PDF output
produced by executing

\begin{itemize}

  \item[] \texttt{pdflatex Assignment\_4\_\emph{YourMacID}}

\end{itemize}

This assignment is due \textbf{Sunday, February 28, 2021 before
  midnight.}  You are allow to submit the assignment multiple times,
but only the last submission will be marked.  \textbf{Late submissions
  and files that are not named exactly as specified above will not be
  accepted!}  It is suggested that you submit your preliminary
\texttt{Assignment\_4\_\emph{YourMacID}.tex} and
\texttt{Assignment\_4\_\emph{YourMacID}.pdf} files well before the
deadline so that your mark is not zero if, e.g., your computer fails
at 11:50 PM on February 28.

\textbf{Although you are allowed to receive help from the
  instructional staff and other students, your submission must be your
  own work.  Copying will be treated as academic dishonesty! If any of
  the ideas used in your submission were obtained from other students
  or sources outside of the lectures and tutorials, you must
  acknowledge where or from whom these ideas were obtained.}

\newpage

\subsection*{Presenting DFAs and NFAs Transition Diagrams}

In this assignment you are asked to present DFAs as transition
diagrams.  You are can do this in one of two ways.

The first way is to present the diagram using the LaTeX graphics
package TikZ.  The TikZ code can either be written by hand or
automatically generated using the finsm system available at
\texttt{http:finsm.io}.

Here are some examples of how it can be used to create
DFA and NFA transition diagrams that appear in the lectures slides:

\begin{center}
\begin{tikzpicture}[shorten >=1pt,node distance=2.5cm,on grid,auto] 
   \node[state, initial, thick] (q_0)   {$q_0$}; 
   \node[state, thick] (q_1) [right=of q_0] {$q_1$}; 
   \node[state, thick] (q_2) [right=of q_1] {$q_2$}; 
   \node[state, accepting, thick] (q_3) [right=of q_2] {$q_3$};
    \path[->, thick, >=stealth] 
    (q_0) edge [bend left, above] node {$a$} (q_1)
          edge [loop, above] node {$b$} (q_2)
    (q_1) edge [bend left, above] node {$a$} (q_2)
          edge [bend left, above] node {$b$} (q_0)
    (q_2) edge [bend left, above] node {$a$} (q_3) 
          edge [bend left, below]  node {$b$} (q_0)
    (q_3) edge [loop, above] node {$a,b$} (q_2); 
\end{tikzpicture}
\end{center}

\begin{center}
\begin{tikzpicture}[shorten >=1pt,node distance=4cm,on grid,auto] 
   \node[state, initial, accepting, thick] (q_0)   {$q_0$}; 
   \node[state, thick] (q_1) [right=of q_0] {$q_1$}; 
   \node[state, thick] (q_2) [below=of q_0] {$q_2$}; 
   \node[state, thick] (q_3) [right=of q_2] {$q_3$};
    \path[->, thick, >=stealth] 
    (q_0) edge [bend left, above] node {1} (q_1)
          edge [bend left, right] node {0} (q_2)
    (q_1) edge [bend left, below] node {1} (q_0)
          edge [bend left, right] node {0} (q_3)
    (q_2) edge [bend left, above] node {1} (q_3) 
          edge [bend left, left]  node {0} (q_0)
    (q_3) edge [bend left, below] node {1} (q_2) 
          edge [bend left, left]  node {0} (q_1);
\end{tikzpicture}
\end{center}

\begin{center}
\begin{tikzpicture}[shorten >=1pt,node distance=1.7cm,on grid,auto] 
   \node[state, initial, thick] (q_0)   {$q_0$}; 
   \node[state, thick] (q_1) [right=of q_0] {$q_1$}; 
   \node[state, thick] (q_2) [right=of q_1] {$q_2$}; 
   \node[state, thick] (q_3) [right=of q_2] {$q_3$};
   \node[state, thick] (q_4) [right=of q_3] {$q_4$};
   \node[state, thick, accepting] (q_5) [right=of q_4] {$q_5$};
    \path[->, thick, >=stealth] 
    (q_0) edge [loop, above] node {0,1} (q_1)
          edge [right, above] node {1} (q_1)
    (q_1) edge [right, above] node {0,1} (q_2)
    (q_2) edge [right, above] node {0,1} (q_3)
    (q_3) edge [right, above] node {0,1} (q_4)
    (q_4) edge [right, above] node {0,1} (q_5);
\end{tikzpicture}
\end{center}

The second way is to take a picture of a hand-written transition
diagram and then embed it into your assignment using the following
LaTeX code:
\begin{verbatim}
\begin{center}
\includegraphics[scale = 0.5]{diagram.jpg}
\end{center}
\end{verbatim}
Please make sure your diagram is legible.

\subsection*{Problems}

\be

  \item \textbf{[10 points]} Construct a deterministic finite
    automaton (DFA) $A$ for the alphabet $\Sigma = \set{0,1}$ such
    that $L(A)$ is the set of all strings $x$ in $\Sigma^*$ for which
    $\#0(x)$ is even and $\#1(x)$ is divisible by 3.  Present $A$ as a
    transition diagram.

  \textcolor{blue}{\textbf{Aidan Mariglia, mariglia, 28/02/21}}\\
  
  \begin{center}
    \begin{tikzpicture}[]
        \node[initial,thick,state] at (-5.5,2.4125) (5c574d8f) {$q_{1}$};
        \node[thick,state] at (-2.475,4.0625) (f5e4e523) {$q_{2}$};
        \node[thick,state] at (1,4) (0554a69a) {$q_{3}$};
        \node[thick,state] at (-3.5,6) (a8100bb0) {$q_{4}$};
        \node[thick,state] at (-4.5,7.75) (df693a97) {$q_{5}$};
        \node[thick,state] at (-5.5,9.5) (552f4a3e) {$q_{6}$};
        \node[thick,state] at (1.75,5.75) (edcc69bb) {$q_{7}$};
        \node[thick,state] at (2.5,7.75) (81c91149) {$q_{8}$};
        \node[thick,accepting,state] at (3.5,9.75) (441c4c5f) {$q_{9}$};
        \node[thick,state] at (-3.425,0.6625) (0607f7de) {$q_{10}$};
        \node[thick,state] at (-0.25,-1.25) (579a937f) {$q_{11}$};
        \node[thick,state] at (3,-3.25) (b06f8842) {$q_{12}$};
        \node[thick,state] at (-4.75,-1) (d1e1261e) {$q_{13}$};
        \node[thick,state] at (-6,-2.75) (efd87ad1) {$q_{14}$};
        \node[thick,state] at (-1.775,-3.3) (54eb5a4a) {$q_{15}$};
        \node[thick,state] at (-2.75,-5.25) (d68779f3) {$q_{16}$};
        \node[thick,state] at (1.75,-5.25) (3966b6b0) {$q_{17}$};
        \node[thick,accepting,state] at (0.5,-7.25) (994bfcd8) {$q_{18}$};
        \path[->, thick, >=stealth]
        (5c574d8f) edge [right,in = -151, out = 29] node {$0$} (f5e4e523)
        (5c574d8f) edge [left,in = 140, out = -40] node {$1$} (0607f7de)
        (f5e4e523) edge [above,in = 140, out = 33] node {$0$} (0554a69a)
        (f5e4e523) edge [right,in = -62, out = 118] node {$1$} (a8100bb0)
        (0554a69a) edge [below,in = -31, out = -154] node {$0$} (f5e4e523)
        (0554a69a) edge [right,in = -113, out = 67] node {$1$} (edcc69bb)
        (a8100bb0) edge [right,in = -60, out = 120] node {$1$} (df693a97)
        (a8100bb0) edge [below,in = -166, out = -18] node {$0$} (edcc69bb)
        (df693a97) edge [right,in = -60, out = 120] node {$1$} (552f4a3e)
        (df693a97) edge [above,in = 170, out = 10] node {$0$} (81c91149)
        (552f4a3e) edge [left,in = 171, out = -113] node {$1$} (a8100bb0)
        (552f4a3e) edge [above,in = 172, out = 10] node {$0$} (441c4c5f)
        (edcc69bb) edge [right,in = -111, out = 69] node {$1$} (81c91149)
        (edcc69bb) edge [above,in = 8, out = 166] node {$0$} (a8100bb0)
        (81c91149) edge [right,in = -117, out = 63] node {$1$} (441c4c5f)
        (81c91149) edge [below,in = -10, out = -170] node {$0$} (df693a97)
        (441c4c5f) edge [right,in = 27, out = -64] node {$1$} (edcc69bb)
        (441c4c5f) edge [below,in = -5, out = -171] node {$0$} (552f4a3e)
        (0607f7de) edge [left,in = 149, out = -31] node {$1$} (579a937f)
        (0607f7de) edge [left,in = 51, out = -129] node {$0$} (d1e1261e)
        (579a937f) edge [left,in = 148, out = -32] node {$1$} (b06f8842)
        (579a937f) edge [left,in = 53, out = -127] node {$0$} (54eb5a4a)
        (b06f8842) edge [right,in = -1, out = 117] node {$1$} (0607f7de)
        (b06f8842) edge [left,in = 58, out = -122] node {$0$} (3966b6b0)
        (d1e1261e) edge [left,in = 99, out = -171] node {$0$} (efd87ad1)
        (d1e1261e) edge [left,in = 142, out = -38] node {$1$} (54eb5a4a)
        (efd87ad1) edge [right,in = -90, out = 22] node {$0$} (d1e1261e)
        (efd87ad1) edge [left,in = 142, out = -38] node {$1$} (d68779f3)
        (54eb5a4a) edge [left,in = 104, out = -151] node {$0$} (d68779f3)
        (54eb5a4a) edge [left,in = 151, out = -29] node {$1$} (3966b6b0)
        (d68779f3) edge [left,in = 148, out = -32] node {$1$} (994bfcd8)
        (d68779f3) edge [right,in = -76, out = 20] node {$0$} (54eb5a4a)
        (3966b6b0) edge [left,in = 90, out = -158] node {$0$} (994bfcd8)
        (3966b6b0) edge [right,in = -14, out = 111] node {$1$} (d1e1261e)
        (994bfcd8) edge [right,in = -90, out = 22] node {$0$} (3966b6b0)
        (994bfcd8) edge [left,in = -65, out = 171] node {$1$} (efd87ad1)
        ;
    \end{tikzpicture}
    \end{center}

  \newpage

  \item \textbf{[10 points]} Let $B$ be the DFA given by the following
    transition diagram:

	\begin{center}
	\begin{tikzpicture}[]
		\node[thick,initial,state] (q0) {$q_{0}$};
		\node[thick,accepting, state] (q1)  [right= of q0]{$q_{1}$};
		\node[thick,state] (q2)  [right= of q1]{$q_{2}$};
		\path[->, thick, >=stealth]
		(q0) edge [bend left,above] node {$1$} (q1)
		edge [loop above]   node         {0} ()
		(q1) edge [bend left,above] node {$0$} (q0)
		edge [bend left,above] node {$1$} (q2)
		(q2) edge [bend left,above] node {$1$} (q1)
		edge [bend left = 40, below] node {$0$} (q0)
		;
	\end{tikzpicture}
        \end{center}

   Prove that $L(B)$ is the language of all binary strings that end
   with an odd number of $1$s.  Hint: Use weak induction on the length
   of the input string, and let $P(n)$ be the statement that, for all
   input strings $w$ with $|w| = n$, the following conditions hold:

   \be

     \item If $\delta^*(q_0,w) = q_0$, then $w$ is $\epsilon$ or ends
       with $0$.

     \item If $\delta^*(q_0,w) = q_1$, then $w$ ends with an odd
       number of $1$s.

     \item If $\delta^*(q_0,w) = q_2$, then $w$ ends
       with an even number of $1$s.

   \ee
	
   \textcolor{blue}{\textbf{Aidan Mariglia, mariglia, 28/02/21}}\\
   

   \begin{proof}
    Let $P(n)$ hold iff statements a, b, c hold for all input strings $w$
    with $|w| = n$.

    \emph{Condition $a$:}
    

    \emph{Base case:} $n = 0$. We must show $P(0)$.\\
    Since we know $\eps$ is the only word with length 0, and following the
    transition diagram for the base case leaves us at $q_0$, $P(0)$ holds.

    \emph{Induction step:} We must show $P(n)$.\\
    As $\delta^*$ is defined for $q_0, q_1$, and $q_2$, all paths
    that transition to $q_0$ require the current character to be $0$.
    Therefore, a word of any length whose final state is $q_0$ must end in a zero,
    therefore, $P(0)$ holds.

    \emph{Condition $b$:}


    \emph{Base case:} $n = 0$. We must show $P(0)$.\\
    If condition b is negated we have:
    If $\delta^*(q_0, w) != q_1$, then w ends with an odd number of
    1s. Since we know with n = 0 the only possible value of w is $\eps$ and,
    $\delta^*(q_0, \eps) = q_0$. So the first half of the condiiton is true.
    $\eps$ does not have an odd number of 1s, so the second half is also true.
    Therefore, $P(0)$ holds.

    \emph{Induciton step:} We must show $P(n)$.\\
    Since any instances of a 0 will "reset" to state $q_0$,
    we can reason about only the trailing ones from input string $w$.
    As previously stated, and section of 1s will always start on state
    $q_0$. We can define the ending substring of ones as being either an
    odd number of ones, $2n+1$, or an even number of ones $2n + 2$. Considering
    an odd number of ones, the first 1 will cause a transition from $q_0$ to $q_1$.
    $q_1$ and $q_2$ are in a loop that will return to its original state for every
    two 1s that it is passed, we can reason that as a result of the loop,
    the remainin $2n$ 1s in the substring will not cause a transition to another
    state after they are all passed, meaning
    that for an ending substring of $2n+1$ 1s, the final state will always be $q_1$.
    Therefore, $P(0)$ holds.

    \emph{Condition $c$:}

    \emph{Base case:} $n = 0$. We must show $P(0)$\\
    Following the same logic presented for the base case of condiiton b,
    we can see that the only possible value of $w$ for $n = 1$ is $\eps$,
    and $w = \eps$ results in a final state of $q_0$. Since
    $\eps$ does not end in an even number of ones, $P(0)$ holds.

    \emph{Induction step:} We must show $P(n)$.
    There are three possible cases for the ending of $w$:
    
    \begin{itemize}
      \item Any number of 0s
      \item An odd number of 1s
      \item An even number of 1s
    \end{itemize}

    Since we have proven that to end on state $q_0$ or $q_1$, you must
    have any number of 0s or an even number of 1s respectfully, we can
    deduce that the only possible end state of an input string $w$ with an
    even number of 1s at its end is $q_2$. Therefore, $P(n)$ holds.

    Q.E.D.

   \end{proof}

\ee

\end{document}


